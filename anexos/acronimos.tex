%%%%%%%%%%%%%%%%%%%%%%%%%%%%%%%%%%%%%%%%%%%%%%%%%%%%%%%%%%%%%%%%%%%%%%%%
% Plantilla TFG/TFM
% Escuela Politécnica Superior de la Universidad de Alicante
% Realizado por: Jose Manuel Requena Plens
% Contacto: info@jmrplens.com / Telegram:@jmrplens
%%%%%%%%%%%%%%%%%%%%%%%%%%%%%%%%%%%%%%%%%%%%%%%%%%%%%%%%%%%%%%%%%%%%%%%%

% Lista de acrónimos (se ordenan por orden alfabético automáticamente)

% La forma de definir un acrónimo es la siguiente:
% \newacronym{id}{siglas}{descripción}
% Donde:
% 	'id' es como vas a llamarlo desde el documento.
%	'siglas' son las siglas del acrónimo.
%	'descripción' es el texto que representan las siglas.
%
% Para usarlo en el documento tienes 4 formas:
% \gls{id} - Añade el acrónimo en su forma larga y con las siglas si es la primera vez que se utiliza, el resto de veces solo añade las siglas. (No utilices este en títulos de capítulos o secciones).
% \glsentryshort{id} - Añade solo las siglas de la id
% \glsentrylong{id} - Añade solo la descripción de la id
% \glsentryfull{id} - Añade tanto  la descripción como las siglas


\newacronym{ndvi}{NDVI}{Índice de Vegetación de Diferencia Normalizada}
\newacronym{tfg}{TFG}{Trabajo Fin de Grado}
\newacronym{rai}{RAI}{Real Academia de Ingeniería}
\newacronym{rf}{RF}{Random Forest}
\newacronym{rfr}{RFR}{Random Forest Regressor}
\newacronym{sar}{SAR}{Synthetic Aperture Radar}
\newacronym{aema}{AEMA}{Agencia Europea de Medio Ambiente}
\newacronym{mmc}{MMC}{Método de Mínimos Cuadrados}
\newacronym{esa}{ESA}{European Space Agency}
\newacronym{ml}{ML}{Machine Learning}
\newacronym{eos}{EOS}{Earth Observing System}
\newacronym{bbch}{BBCH}{Biologische Bundesanstalt, Bundessortenamt und CHemische Industrie}
\newacronym{grd}{GRD}{Ground Range Detected}
\newacronym{sst}{SST}{Señales, Sistemas y Telecomunicación}
\newacronym{ua}{UA}{Universidad de Alicante}
\newacronym{ekf}{EKF}{Extended Kalman Filter}
\newacronym{rmse}{RMSE}{Root-Mean-Square Error}
\newacronym{mae}{MAE}{Mean Absolute Error}
\newacronym{pdf}{PDF}{Probability Density Function}