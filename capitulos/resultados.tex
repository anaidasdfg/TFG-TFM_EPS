%%%%%%%%%%%%%%%%%%%%%%%%%%%%%%%%%%%%%%%%%%%%%%%%%%%%%%%%%%%%%%%%%%%%%%%%
% Plantilla TFG/TFM
% Escuela Politécnica Superior de la Universidad de Alicante
% Realizado por: Jose Manuel Requena Plens
% Contacto: info@jmrplens.com / Telegram:@jmrplens
%%%%%%%%%%%%%%%%%%%%%%%%%%%%%%%%%%%%%%%%%%%%%%%%%%%%%%%%%%%%%%%%%%%%%%%%

\chapter{Resultados}
\label{resultados}
\section{Método por parcelas}
\subsection{Optimización}
\par La optimización de los datos de entrada tiene su base en la elección de las variables de entrada y de los sets de parcelas de entrenamiento y test. Los mejores resultados para todos los casos el uso de un conjunto de 6 parcelas para entrenamiento y 1 para la evaluación, y como datos de entrada al sistema todos los mencionados anteriormente: media por día y parcela de VV, VH y ratio VH/VV y la desviación estándar de cada uno de ellos. En cuanto a la optimización del regresor, se basa en la determinación del número de árboles que lo componen. La relación entre el número de árboles para un modelo y el coeficiente de determinación es un buen descriptor para la elección de este parámetro, como se presenta en la imagen de ejemplo \ref{fig:opt_parcl} con el caso de salida \gls{bbch}. En ella, se puede ver que una vez alcanzado cierto nivel de coeficiente, la mejora de este en relación al aumento del número de árboles no es significativa con respecto al costo computacional y a la complejidad del sistema que se crea. 
\begin{figure}[h]
    \centering
    \includegraphics[height=9cm]{archivos/tfg/Mean/opt_tree_bbch_mean} 
    \caption{Optimización del número de árboles para \gls{rfr} en el modelo de salida \gls{bbch}\label{fig:opt_parcl}}
    
\end{figure}

\par Finalmente, los parámetros utilizados en este método para cada caso se presentan en la tabla \ref{tab:opt_parcl}, donde se pueden ver: la parcela utilizada para el periodo de test del modelo; siendo el resto de parcelas utilizadas en el entrenamiento, y el número de árboles óptimo para \gls{rfr}, de acuerdo con la evolución del coeficiente de determinación para cada caso.

\begin{table}[h]
\centering
\begin{tabular}{l|ccc}
                  & \gls{bbch}     & Altura   & \gls{bbch}\&Altura \\ \hline
Parcela de test   & `Mínima' & `Puntal' & `Mínima'     \\
Número de árboles & 42       & 77       & 54          
\end{tabular}
\caption{Parámetros de optimización de entrada y modelo
\label{tab:opt_parcl}}
\end{table}

\par Como se puede observar, los 3 casos de este método constan de un número óptimo de árboles de, al menos, el mismo orden. Cabe destacar que el aumento de complejidad en el modelo, sobre todo en el caso de la altura como salida del sistema.  
\subsection{Salidas del modelo}
\subsubsection{Salidas de función de densidad de probabilidad}
Como ya se ha comentado, las salidas del modelo son por defecto predicciones únicas, pero el marco de trabajo anterior demanda salidas en formato \gls{pdf} para ser integradas con el resto del modelo. En las figuras \ref{fig:pdf_b} (modelo de salida \gls{bbch}), \ref{fig:pdf_h} (modelo de salida altura) y \ref{fig:pdf_bh} (modelo de ambas salidas) se puede apreciar cómo son algunas de estas salidas, siendo ejemplos extraídos a partir de los mismos datos para los 3 casos estudiados. 
\\
% Dos figuras sueltas debajo de otra
\begin{figure}[h]
\centering
\includegraphics[width=0.85\linewidth]{archivos/tfg/Mean/TEST_PARC_PDF}
\captionof{figure}{Ejemplo de salida \gls{pdf} normalizada del modelo para estimación de \gls{bbch}.\label{fig:pdf_b}}
\end{figure}
\begin{figure}[h]
\centering
\includegraphics[width=0.85\linewidth]{archivos/tfg/Mean/TEST_PARC_PDF_H}
\captionof{figure}{Ejemplo de salida \gls{pdf} normalizada del modelo para estimación de la altura. \label{fig:pdf_h}}
\end{figure}

\par Las \gls{pdf}s se presentan normalizadas con respecto a la salida con mayor probabilidad. En estos ejemplos se puede ver como, aunque hay una salida que predomina con respecto al resto, no se excluyen las demás soluciones posibles. Esto facilita la integración con el modelo de predicción temporal ya que se pueden combinar ambas salidas \gls{pdf} para ver dónde coinciden y con qué probabilidad. En general, los resultados de la predicción temporal son bastante certeros, con oportunidad de fallo para ajustes finos en cultivos que hayan podido sufrir retrasos o adelantos en su desarrollo típico. Es ahí donde la contribución de este modelo es importante, aportando información extra en tiempo real y creando una predicción de cuáles serían los posibles estados de desarrollo en los que se encuentra el cultivo según la información actual de satélite. 
\\
\par Las 3 figuras \ref{fig:pdf_b}, \ref{fig:pdf_h} y \ref{fig:pdf_bh} han sido generadas con los mismos datos de entrada, es decir, los mismos datos de satélite en la misma parcela y fecha, por lo que las salidas para los modelos generados de estimación de \gls{bbch} (Figura \ref{fig:pdf_b}) y altura (Figura \ref{fig:pdf_h}) como modelos independientes se pueden comparar con las salidas del modelo de estimación de ambas (Figura \ref{fig:pdf_bh}). La principal diferencia que encontramos entre los mismos tipos de datos de salida para cada modelo es que, en general, las salidas individuales presentan una estimación principal con una diferencia de probabilidad mucho mayor con respecto al resto que las estimaciones del modelo de doble salida. El rango para cada salida se mantiene bastante similar en ambos modelos, además del valor con la probabilidad más alta, aunque la disminución de diferencias con las demás estimaciones, sobre todo en el caso de la altura (\ref{fig:sub2}), indica que ese sistema es menos preciso y estable. El hecho de que sea la salida de la altura la que se vea más perjudicada en este modelo de dos salidas puede deberse a que la optimización de este, en cuanto a número de árboles del regresor, es más similar, y por lo tanto más beneficiosa, con respecto al caso individual de estimación de \gls{bbch}.

% Una figura con dos imágenes
\begin{figure}[H]
\centering
\begin{subfigure}{0.9\textwidth}
  \centering
  \includegraphics[width=0.95\linewidth]{archivos/tfg/Mean/TEST_PARC_PDF_BH_B}
  \caption{Ejemplo del modelo de doble salida: \gls{pdf} estimación de \gls{bbch}\label{fig:sub1}}
\end{subfigure}
\begin{subfigure}{0.9\textwidth}
  \centering
  \includegraphics[width=0.95\linewidth]{archivos/tfg/Mean/TEST_PARC_PDF_BH_H}
  \caption{Ejemplo del modelo de doble salida: \gls{pdf} estimación de altura\label{fig:sub2}}
\end{subfigure}
\caption{Ejemplo de salidas \gls{pdf} normalizadas del modelo para estimación de \gls{bbch} y la altura. \label{fig:pdf_bh}}
\end{figure}

\subsubsection{Salidas de valor único}
\par Para continuar con la presentación de los resultados obtenidos, se ilustran en las figuras \ref{fig:comp_b}, \ref{fig:comp_h} y \ref{fig:comp_bh} las comparaciones de las soluciones únicas obtenidas (salida con máxima probabilidad) y los datos de tierra tomados correspondientes para la parcela de test en 2018: \gls{bbch} general, máxima y mínima en la parcela, casos individual (\ref{fig:comp_b}) y de doble salida (\ref{fig:sub_c1}), o la altura general del cultivo, también para ambos casos (figuras \ref{fig:comp_h} y \ref{fig:sub_c2}, respectivamente). Estas salidas, aunque carecen de la información completa, son las más sencillas de comparar y evaluar con los resultados reales, ya que estos también son un único valor.  
\\

% Dos figuras sueltas debajo de otra
\begin{figure}[h]
\centering
\includegraphics[width=\linewidth]{archivos/tfg/Mean/TEST_PARC_FINAL}
\captionof{figure}{Ejemplo de salida \gls{pdf} normalizada del modelo para estimación de \gls{bbch}.\label{fig:comp_b}}
\end{figure}
\begin{figure}[h]
\centering
\includegraphics[width=\linewidth]{archivos/tfg/Mean/TEST_PARC_FINAL_H}
\captionof{figure}{Ejemplo de salida \gls{pdf} normalizada del modelo para estimación de la altura. \label{fig:comp_h}}
\end{figure}

\par Comenzando por los modelos de predicción de \gls{bbch} representados en las figuras \ref{fig:comp_b} y \ref{fig:sub_c1}, se pueden observar estimaciones y errores muy parecidos en ambos modelos. Estos son fácilmente comparables ya que la parcela de test coincide para los dos casos. Ambos modelos, presentan estimaciones que, en su mayoría, están fuera de los límites marcados por las líneas de \gls{bbch} máximas y mínimas medidas en campo. En general, se ve una tendencia creciente a lo largo de la variable estimada entorno a los valores reales, aunque con 2 o 3 zonas con picos de error mayor en las que el sistema predice un estado fenológico mayor al real. De hecho, el primer pico en torno a los días 30-40 después de la siembra está presente tanto en la estimación de \gls{bbch} como de la altura en las figuras \ref{fig:comp_h} y \ref{fig:sub_c2}. Teniendo en cuenta que los valores de entrada son iguales para todos los casos, se puede atribuir este fenómeno a anomalías presentes en los datos de satélite, los cuales, en  una etapa concreta del desarrollo del cultivo, pueden presentar valores muy similares a los obtenidos para etapas finales, y por ello cometer un error de predicción bastante grande. Este error sería corregible bien detectando la anomalía y eliminando o modificando esta parte de los datos de entrada, tanto en entrenamiento como en test o futuros usos, o bien comprobando que el modelo está generando predicciones correctas con menor densidad de probabilidad, estando representadas en las salidas de \gls{pdf}, y pudiendo ser consideradas sin un procesamiento extra de los datos, simplemente al contrastarlos con los resultados del modelo temporal, ya que es una diferencia de etapas muy grande y en este modelo no se daría un error de etapa de tal dimensión. 
\\
\par Visualizando las salidas para la altura, se puede también comparar, aunque no directamente, el funcionamiento de ambos modelos. En este caso las parcelas de test no son las mismas debido a la optimización para cada modelo, por lo que la comparación simplemente visual entre las dos representaciones es menos intuitiva. Aún así, se aprecian coincidencias como el error en la etapa cercana a 40 días, como se ha mencionado antes, o la estimación de mayor altura para los días de 0 a 10 después de la siembra. En general, ambas representaciones tienen también una tendencia creciente bastante similar a la real, con aparentemente mejor estimación para las etapas finales en el modelo de salida única de altura, ya que en la figura \ref{fig:comp_h} se observa una estimación menor constante para todas las etapas desde el día 80 tras la siembra hasta la cosecha (día 120 aproximadamente).

% Una figura con dos imágenes
\begin{figure}[H]
\centering
\begin{subfigure}{\textwidth}
  \centering
  \includegraphics[width=0.95\linewidth]{archivos/tfg/Mean/TEST_PARC_FINAL_BH}
  \caption{Ejemplo del modelo de doble salida: \gls{pdf} estimación de \gls{bbch}\label{fig:sub_c1}}
\end{subfigure}
\begin{subfigure}{\textwidth}
  \centering
  \includegraphics[width=0.95\linewidth]{archivos/tfg/Mean/TEST_PARC_FINAL_BH_H}
  \caption{Ejemplo del modelo de doble salida: \gls{pdf} estimación de altura\label{fig:sub_c2}}
\end{subfigure}
\caption{Ejemplo de salidas \gls{pdf} normalizadas del modelo para estimación de \gls{bbch} y la altura. \label{fig:comp_bh}}
\end{figure}
\subsection{Evaluación de resultados}

%%%%%%%%%%%%%%%%%%%
\section{Método por píxeles} 
\subsection{Optimización}
\par La optimización de los datos de entrada para este método se realiza de igual manera que el anterior, coincidiendo los mejores resultados para todos los casos en el uso de un conjunto de 6 parcelas para entrenamiento y 1 para la evaluación y de igual manera para los datos de entrada, aunque la desviación estándar no aporta una gran mejora para este método. En cuanto a la optimización del regresor, se utiliza también la relación entre el número de árboles para un modelo y el coeficiente de determinación, presentado en la imagen de ejemplo \ref{fig:opt_pixl} con el caso de salida \gls{bbch}. En ella, se puede ver que una vez alcanzado cierto nivel de coeficiente, la mejora de este en relación al aumento del número de árboles no es significativa con respecto al costo computacional y a la complejidad del sistema que se crea. 
\begin{figure}[h]
    \centering
    \includegraphics[height=9cm]{archivos/tfg/Pixel/opt_tree_bbch_pixel} 
    \caption{Optimización del número de árboles para \gls{rfr} en el modelo de salida \gls{bbch}}
    \label{fig:opt_pixl}
\end{figure}

\par Finalmente, los parámetros utilizados en este método para cada caso se presentan en la tabla \ref{tab:opt_pixl}, donde se pueden ver: la parcela utilizada para el periodo de test del modelo; siendo el resto de parcelas utilizadas en el entrenamiento, y el número de árboles óptimo para \gls{rfr}, de acuerdo con la evolución del coeficiente de determinación para cada caso.

\begin{table}[h]
\centering
\begin{tabular}{l|ccc}
                  & gls{bbch}     & Altura   & gls{bbch}\&Altura \\ \hline
Parcela de test   & `Mínima' & `Mínima' & `Mínima'     \\
Número de árboles & 46       & 48       & 47          
\end{tabular}
\caption{Parámetros de optimización de entrada y modelo \label{tab:opt_pixl}}
\end{table}

\par Como se puede observar, los 3 casos coinciden en el set de parcelas de entrenamiento y test con el que se obtienen mejores resultados. Probablemente esto se debe a anomalías en el desarrollo de algunas parcelas, las cuales, si se toman como set de test, el modelo no las habría podido tener en cuenta en el aprendizaje y el error en la predicción sería mayor. Los 3 casos de este método constan de un número óptimo de árboles muy similares, con una diferencia de 1 y 2 con respecto al caso menor. Cabe destacar que el aumento de complejidad, aunque escaso, en el modelo de salida de la altura, como ocurre para el método anterior. Además, tanto para la metodología por píxeles como por parcelas, se obtiene un número óptimo de árboles para el caso de 2 salidas intermedio a los óptimos para las mismas salidas en modelos independientes. Esto se debe a una compensación en la estimación de ambas variables, para que una variable sea óptima sin que su mejora sea a costa de la otra se llega a un valor intermedio en el que ninguna de las salidas está totalmente optimizada pero tienen el mejor resultado del sistema compartido completo. 


\subsection{Salidas del modelo}
\subsubsection{Salidas de función de densidad de probabilidad}
\subsubsection{Salidas de valor único}
\subsection{Evaluación de resultados}
%%%%%%%%%%%%%%%%%%%%%%%%%%%%%%%%%%%%%%%%%%%
%%%%%%%%%%%%%%%%%%%%%%%%%%%%%%%%%%%%%%%%%%%
\section{Comparativa de métodos}
\par La evaluación general de resultados realizada es la comparación entre los dos métodos de procesamiento de datos de entrada mencionados: a nivel de parcela o de pixel. En las siguientes tablas se pueden ver la evaluación de los resultados, divididos en entradas a nivel de parcela ~\ref{tab:errorpc} y a nivel de pixel ~\ref{tab:errorpx}, según los índices estadísticos de \gls{mae}, \gls{rmse} y el coeficiente de determinación ($R^2$). Ambos conjuntos corresponden al mejor caso de cada método, esto es, habiendo seleccionado sus parámetros de entrada óptimos en cuanto a número de variables y set de parcelas de entrenamiento, y habiendo optimizado también los parámetros del regresor.

\begin{table}[h]
\centering
\begin{tabular}{lccc}
\multicolumn{4}{c}{Datos de entrada a nivel de parcela}                            \\ \hline \hline
\multicolumn{1}{l|}{}                            & BBCH  & Altura & BBCH\&Altura \\ \cline{2-4} 
\multicolumn{1}{l|}{$R^2$}                       & 0.74  & 0.65   & 0.70 \\
\multicolumn{1}{l|}{\gls{rmse}} 				 & 15.39 & 17.87  & 15.69 \\
\multicolumn{1}{l|}{\gls{mae}}  				 & 11.13 & 14.21  & 12.63       
\end{tabular}
\caption{Índices estadísticos de las predicciones con datos de entrada a nivel de parcela. \label{tab:errorpc}}
\end{table}

\begin{table}[h]
\centering
\begin{tabular}{lccc}
\multicolumn{4}{c}{Datos de entrada a nivel de pixel}                            \\ \hline \hline
\multicolumn{1}{l|}{}                            & BBCH  & Altura & BBCH\&Altura \\ \cline{2-4} 
\multicolumn{1}{l|}{$R^2$}                       & 0.45  & 0.52   & 0.45         \\
\multicolumn{1}{l|}{\gls{rmse}} 				 & 22.23 & 19.89  & 21.17        \\
\multicolumn{1}{l|}{\gls{mae}}  				 & 17.26 & 15.64  & 16.58       
\end{tabular}
\caption{Índices estadísticos de las predicciones con datos de entrada a nivel de pixel.\label{tab:errorpx}}
\end{table}

\par Como se puede observar, para las 3 posibles salidas para las que se han diseñado los modelos, el método que emplea datos de entrada a nivel de parcela consigue notablemente mejores resultados. Se obtienen valores de coeficiente de determinación más cercanos a 1, valor ideal, apreciándose en el modelo de predicción de \gls{bbch} una mejora para el caso de parcelas de un 64.4\% con respecto al de pixel. Lo mismo ocurre en los índices de error, cuyo valor óptimo es 0: se obtienen valores inferiores para el método a nivel de parcela en todos los casos estudiados.
\\
\par Los mejores resultados obtenidos para el procesamiento a nivel de parcelas pueden recaer en que todos los datos recogidos de verdad de tierra se presentan por parcelas, por lo que no existen unos datos para contrastar cada pixel extraído de la información de satélite con el terreno. Dentro de cada parcela el cultivo no tiene porqué desarrollarse de manera homogénea, pero la información recibida de este consta del valor de \gls{bbch} (con el requisito de ser el correcto para al menos el 50\% del cultivo) y la altura generalizados por parcela, además de los máximos y mínimos de cada uno, por lo que las variaciones que se puede hallar en distintas zonas del cultivo no se tienen en cuenta. Por ello, el método de pixel implementado asigna un solo valor de salida para conjuntos de píxeles de la misma parcela, los cuales pueden estar representando unos niveles de \gls{bbch} o altura distintos, y esto conlleva un ajuste del modelo erróneo y, por tanto, peores índices estadísticos de evaluación. Aún así, este método no ha sido descartado, ya que el análisis a nivel de pixel es de gran interés para detectar esas variaciones de desarrollo de un cultivo dentro de una misma parcela. 