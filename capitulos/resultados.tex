%%%%%%%%%%%%%%%%%%%%%%%%%%%%%%%%%%%%%%%%%%%%%%%%%%%%%%%%%%%%%%%%%%%%%%%%
% Plantilla TFG/TFM
% Escuela Politécnica Superior de la Universidad de Alicante
% Realizado por: Jose Manuel Requena Plens
% Contacto: info@jmrplens.com / Telegram:@jmrplens
%%%%%%%%%%%%%%%%%%%%%%%%%%%%%%%%%%%%%%%%%%%%%%%%%%%%%%%%%%%%%%%%%%%%%%%%

\chapter{Resultados}
\label{resultados}
\par Los resultados generales obtenidos para cada uno de los casos mencionados anteriormente, una vez optimizados los datos utilizados y los parámetros del regresor, se pueden considerar bastante buenos. Las evaluaciones de estos resultados se realizan con indicadores de error como el la raíz del error cuadrático medio (\gls{rmse}) o el error al cuadrado ($R^2$) para las predicciones del set de test, siendo las salidas del modelo una predicción única, y no las probabilidades para cada salida. 

\section{Resultados para método por parcelas}
\section{Resultados para método por píxeles} 
