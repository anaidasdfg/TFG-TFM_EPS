%%%%%%%%%%%%%%%%%%%%%%%%%%%%%%%%%%%%%%%%%%%%%%%%%%%%%%%%%%%%%%%%%%%%%%%%
% Plantilla TFG/TFM
% Escuela Politécnica Superior de la Universidad de Alicante
% Realizado por: Jose Manuel Requena Plens
% Contacto: info@jmrplens.com / Telegram:@jmrplens
%%%%%%%%%%%%%%%%%%%%%%%%%%%%%%%%%%%%%%%%%%%%%%%%%%%%%%%%%%%%%%%%%%%%%%%%

\chapter{Resultados}
\label{resultados}
\par Los resultados generales obtenidos para cada uno de los casos mencionados anteriormente, una vez optimizados los datos utilizados y los parámetros del regresor, se pueden considerar bastante buenos. Las evaluaciones de estos resultados se realizan con indicadores de error como la media del error absoluto (\gls{mae}), correpondiente a la fórmula \ref{eq:MAE}, la raíz del error cuadrático medio (\gls{rmse}), fómula \ref{eq:RMSE} o el coeficiente de determinación ($R^2$), fórmula \ref{eq:R2} para las predicciones del set de test, siendo las salidas del modelo una predicción única, y no las probabilidades para cada salida.

\begin{equation}\label{eq:MAE}
\gls{mae} = \frac{\sum_{i=1}^{n} \left | y_i -x_i \right |}{n}
\end{equation} 
\begin{equation}\label{eq:RMSE}
\gls{rmse} = \sqrt{\frac{\sum_{i=1}^{n}(y_i- x_i )^2}{n}}
\end{equation}
\begin{equation}\label{eq:R2}
R^2 = \frac{\sigma^2_{XY}}{\sigma^2_X\sigma^2_Y}
\end{equation}

\section{Resultados para método por parcelas}
\section{Resultados para método por píxeles} 
