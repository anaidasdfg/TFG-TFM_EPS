 %%%%%%%%%%%%%%%%%%%%%%%%%%%%%%%%%%%%%%%%%%%%%%%%%%%%%%%%%%%%%%%%%%%%%%%%
% Plantilla TFG/TFM
% Escuela Politécnica Superior de la Universidad de Alicante
% Realizado por: Jose Manuel Requena Plens
% Contacto: info@jmrplens.com / Telegram:@jmrplens
%%%%%%%%%%%%%%%%%%%%%%%%%%%%%%%%%%%%%%%%%%%%%%%%%%%%%%%%%%%%%%%%%%%%%%%%

\chapter{Metodología}
\label{metodologia}
\par El desarrollo de este \gls{tfg} va a consistir en la creación de un script de Python que tenga como entradas los datos de \gls{sar} de Sentinel-1, y como salida la función de densidad de probabilidad para la \gls{bbch} y/o altura de la planta, utilizando \gls{rfr} para realizar esta estimación. 
\\
\par Los datos de entrada de satélite de los que disponemos constan de las 7 parcelas de arroz sobre las que se trabaja: Calogne, Ermita, Puntal, Mínima, Puebla, Reboso y Vega, y los valores en dB del coeficiente de backscattering para polarizaciones VV y VH, obtenidos de las imágenes \gls{sar}, con periodo de revista de 6 días. Para realizar el modelo de observación necesitamos contrastar con datos reales de los cultivos utilizados, por lo que se dispone también de los siguientes datos de las 7 parcelas para los años 2017 y 2018: su posición geográfica, área, días de siembra y de cosecha, producción, \gls{bbch} total, mínima y máxima, la altura media del cultivo y los días del año para los cuales se han tomado estos datos. De todos ellos, los más relevantes para este estudio son la \gls{bbch} por parcela, la altura del cultivo y los días de siembra, cosecha y de toma de datos, los cuáles no tienen porqué coincidir con el periodo de revista de los satélites de Sentinel-1.
\\
\par  El procesamiento que los datos de satélite necesitan para la elaboración del modelo es distinto dependiendo del método utilizado. En este proyecto se realizan pruebas para la estimación de 3 casos de salidas distintas: \gls{bbch}, altura (cm) y \gls{bbch} junto a altura (cm). Además, cada uno de estos casos se evalúa utilizando los datos de entrada a nivel de media por parcela y a nivel de pixel, desconociendo a priori qué método obtiene mejores resultados. Para ambos métodos, el procesamiento de los datos comienza restringiendo la información al periodo que nos interesa: desde el día de siembra hasta el de cosecha. A continuación, se ajustan los días de los que se tiene información realizando una interpolación para los datos de \gls{bbch} y/o altura, haciéndolos coincidir con las fechas de toma de datos del satélite, cuyos datos no deben interpolarse ya que su evolución no es tan creciente lineal como la altura o la fenología. El siguiente procesamiento se da únicamente para el método a nivel de parcela: se realiza la media y la desviación estándar de los datos que vamos a utilizar como entradas del sistema (VV en dB, VH en dB, y el ratio entre ambos, VH-VV, también en dB). Para finalizar la preparación de los datos, se dividen estos en sets de entrenamiento y de test. La división se realiza por parcelas completas, para que sea más sencillo y completo su entrenamiento y posterior visualización. Para todos los casos se reservan 6 parcelas de entrenamiento y 1 de test de resultados. Los datos reales con los que se va a entrenar y examinar el modelo se preparan con la interpolación mencionada anteriormente y la división de parcelas que sigue los mismos requisitos que los datos de satélite. La única adaptación extra que tienen estos datos se da en el caso de nivel de pixel: como no contamos con la información medida en tierra a ese nivel de \gls{bbch} ni de altura, los valores generales para cada parcela interpolados deben ser asignados a cada uno de los píxeles correspondientes de esa parcela, es decir, todos los píxeles tendrán el mismo valor de salida para una misma parcela y día. Una interpretación gráfica aproximada de este procesamiento se puede ver en la figura *INSERTAR CROQUIS*