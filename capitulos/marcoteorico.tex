%%%%%%%%%%%%%%%%%%%%%%%%%%%%%%%%%%%%%%%%%%%%%%%%%%%%%%%%%%%%%%%%%%%%%%%%
% Plantilla TFG/TFM
% Escuela Politécnica Superior de la Universidad de Alicante
% Realizado por: Jose Manuel Requena Plens
% Contacto: info@jmrplens.com / Telegram:@jmrplens
%%%%%%%%%%%%%%%%%%%%%%%%%%%%%%%%%%%%%%%%%%%%%%%%%%%%%%%%%%%%%%%%%%%%%%%%

\chapter{Marco Teórico}
\label{marcoteorico}
\par A continuación se expone la teoría necesaria para la comprensión de este \gls{tfg}, ampliando la información ya presentada en el capítulo \ref{introduccion}.
\section{Técnicas de regresión y machine learning}
\par Las técnicas de regresión son todas aquellas técnicas que buscan la relación de una variable dependiente con una o más variables independientes mediante la estimación de su función de regresión, función para la cual se promedian todos los valores de la variable dependiente para unos valores fijos de las variables independientes. Además, también se tiene en cuenta la varianza de la variable dependiente para estos mismos valores, pudiendo ser estudiada también mediante su distribución de probabilidad. Esta varianza indica la fiabilidad de nuestras estimaciones o el "ruido" en las medidas de la variable dependiente. 
\\
\par El caso más sencillo de regresión es el que solo tenemos una variable dependiente y otra independiente, este caso se conoce como regresión lineal, ya que la función de regresión estimada se corresponde a la ecuación de una recta. Los datos que obtenemos para la variable dependiente que vamos a relacionar tienen, aparte de la componente lineal, una componente aleatoria de ruido que puede deberse a distintos fenómenos como la precisión mínima del instrumento de medida, el ruido que este mismo general en la medida o contribuciones de fuentes externas, consideradas como ruido también. 
\\
\par *Introducir aquí parte analítica regresión lineal*
\\ 
\par Cuando introducimos mayor número de variables independientes, la regresión es no lineal, ya que la respuesta de la variable dependiente varía en función a la contribución de cada una de esas dependencias, por lo que la función de regresión presentará mayor complejidad. Cualquier variable independiente que tenga relación con la dependiente es útil en mayor o menor medida pero siempre proporciona información aunque su varianza sea muy grande o su contribución relativamente pequeña. Cualquier tipo de información extra proporciona un ajuste a la estimación final positivo si esta se ha modelado correctamente. 
\\
\par *Introducir aquí parte analítica regresión no lineal*
\\
\par Las técnicas de regresión proporcionan una estimación útil para realizar predicciones, por lo que están relacionados con el aprendizaje automático o machine learning. El machine learning es un tipo de inteligencia artificial, que se caracteriza por la generación de un modelo estimado de manera automática por un computador. Como se puede ver, el objetivo de este aprendizaje y las técnicas de regresión coinciden tienen gran parte de su desarrollo en común. 
\\
\par Algunos de los modelos de machine learning son: 
\section{Teledetección}
\section{Estimación de parámetros físicos de cultivos mediante regresión}
 