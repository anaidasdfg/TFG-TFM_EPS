%%%%%%%%%%%%%%%%%%%%%%%%%%%%%%%%%%%%%%%%%%%%%%%%%%%%%%%%%%%%%%%%%%%%%%%%
% Plantilla TFG/TFM
% Escuela Politécnica Superior de la Universidad de Alicante
% Realizado por: Jose Manuel Requena Plens
% Contacto: info@jmrplens.com / Telegram:@jmrplens
%%%%%%%%%%%%%%%%%%%%%%%%%%%%%%%%%%%%%%%%%%%%%%%%%%%%%%%%%%%%%%%%%%%%%%%%

\chapter{Marco Teórico}
\label{marcoteorico}
\par A continuación se expone la teoría necesaria para la comprensión de este \gls{tfg}, ampliando la información ya presentada en el capítulo \ref{introduccion}.
\section{Técnicas de regresión y machine learning}
\par Las técnicas de regresión son todas aquellas técnicas que buscan la relación de una variable dependiente con una o más variables independientes mediante la estimación de su función de regresión. Para ello se consideran y ponderan todos los valores de la variable dependiente para unos valores fijos de las variables independientes. Además, en estos análisis, también se tiene en cuenta la varianza de la variable dependiente para estos mismos valores, pudiendo ser estudiada también mediante su distribución de probabilidad. Esta varianza indica la fiabilidad de nuestras estimaciones o el "ruido" en las medidas de la variable dependiente. 
\\
\par El caso más sencillo de regresión es en el que solo tenemos una variable dependiente y otra independiente, este caso se conoce como regresión lineal simple, ya que la función de regresión estimada se corresponde a una ecuación lineal de una recta. Los datos que obtenemos para la variable dependiente que vamos a relacionar tienen, aparte de las componentes lineales, una componente aleatoria de ruido que puede deberse a distintos fenómenos como la precisión mínima del instrumento de medida, el ruido que este mismo general en la medida o contribuciones de fuentes externas, consideradas como ruido también. Esta función de regresión es frecuentemente estimada mediante el \gls{mmc}. También existe la regresión lineal múltiple, que funciona de la misma manera pero con mayor número de variables independientes, por lo que en lugar de una recta, la función de regresión representa un plano en el que coinciden N dimensiones, siendo N el número de variables independientes total. 
\\
\par *Introducir aquí parte analítica regresión lineal*
\\ 
\par Cuando la función de regresión no es una función lineal, la regresión es no lineal, ya que la respuesta de la variable dependiente puede ser exponencial, logarítmica o polinomial, entre otras, por lo que la función de regresión presentará mayor complejidad. Aquí también es común utilizar el \gls{mmc} o la regresión segmentada, que ajusta como regresión lineal segmentos de la original no lineal.
\\
\par Cualquier variable independiente que tenga relación con la dependiente es útil en mayor o menor medida pero siempre proporciona información aunque su varianza sea muy grande o su contribución relativamente pequeña. Cualquier tipo de información extra proporciona un ajuste a la estimación final positivo si esta se ha modelado correctamente. 
\\
\par *Introducir aquí parte analítica regresión no lineal*
\\
\par A parte de las regresiones lineales y no lineales mencionados, también encontramos otros métodos de regresión como son los mínimos errores absoluto (bastante similar al \gls{mmc}), la regresión no paramétrica o la regresion lineal bayesiana.

\par Las técnicas de regresión proporcionan una estimación útil para realizar predicciones, por lo que están relacionados con el aprendizaje automático o machine learning. El machine learning es un tipo de inteligencia artificial, que se caracteriza por la generación de un modelo estimado de manera automática por un computador. Como se puede ver, el objetivo de este aprendizaje y las técnicas de regresión coinciden y tienen gran parte de su desarrollo en común. 
\\
\par Los modelos empleados en machine learning son numerosos, y su clasificación se puede realizar dependiendo de su algoritmo de aprendizaje,: 
\begin{itemize}
	\item Aprendizaje supervisado: tanto las entradas al modelo de aprendizaje como las salidas están previamente definidas. Se realiza un entrenamiento en el que se utilizan las entradas con sus correspondientes salidas para elaborar el modelo. Una vez suficientemente entrenado, este puede obtener salidas previamente desconocidas a partir de 
\end{itemize}
\section{Teledetección}
\section{Estimación de parámetros físicos de cultivos mediante regresión}
 