%%%%%%%%%%%%%%%%%%%%%%%%%%%%%%%%%%%%%%%%%%%%%%%%%%%%%%%%%%%%%%%%%%%%%%%%
% Plantilla TFG/TFM
% Escuela Politécnica Superior de la Universidad de Alicante
% Realizado por: Jose Manuel Requena Plens
% Contacto: info@jmrplens.com / Telegram:@jmrplens
%%%%%%%%%%%%%%%%%%%%%%%%%%%%%%%%%%%%%%%%%%%%%%%%%%%%%%%%%%%%%%%%%%%%%%%%

\chapter{Conclusiones}
\label{conclusiones}
\par A rasgos generales, este \gls{tfg} ha cumplido con los objetivos propuestos: se ha desarrollado un modelo de observación para la estimación del desarrollo de los cultivos utilizando imágenes \gls{sar} de satélite y un modelo regresor de aprendizaje automático, \gls{rfr}. Además de cumplimentar el objetivo general de obtener un modelo, se han obtenido resultados considerablemente útiles para la integración con el modelo de predicción temporal anterior perteneciente al marco de trabajo marcado. Se han llegado a estos resultados gracias a la diversidad de pruebas realizadas tanto en metodología como en variables a estimar, lo cual ha aportado riqueza en las evaluaciones de resultados. Aunque, en general, los resultados de las estimaciones a nivel de media hayan obtenido mejor evaluación, la metodología de pixel aporta gran información sobre los rangos de la parcela y, en un futuro estudio, puede ser muy interesante para diferenciar zonas dentro de una misma parcela con distinto nivel de desarrollo, y poder detectar anomalías y actuar para corregirlas. 
\\
\par Dentro de este \gls{tfg} se pueden ampliar algunos campos para profundizar en una línea de trabajo futura o, simplemente,  para obtener mejores resultados en la investigación actual. Un ejemplo de esto sería la optimización de generación del modelo, ya que aquí se han tenido en cuenta 2 parámetros principales de \gls{rfr}, pero python permite la variación de muchos otros más complejos y menos intuitivos que se han mantenido por defecto y pueden llevar a un estudio completo. Otra implementación que se podría llevar a cabo para la mejora del modelo sería la obtención de más datos de entrada, ya fueran datos reales obtenido por el paso de más tiempo o datos estimados a partir de los ya disponibles. Esto haría la etapa de aprendizaje más completa y generalizaría el modelo para distintos comportamientos de desarrollo. Siguiendo en esta línea, un caso ideal de implementación sería la generación de modelos de observación y predicción temporal enfocados a cada parcela individualmente. La generación de modelos en este caso sería mucho más personalizada, ajustándose a las características propias de cada parcela, las cuales serían más estable que los modelos generados para parcelas distintas. Se ha mencionado que sería un caso ideal ya que para su implementación útil se necesitarían una cantidad de datos sobre una sola parcela mucho mayor de la que disponemos actualmente, aunque también su modelado sería menos compleja, debido a la semejanza que habría en todos los casos. 
\\
\par Dejando a un lado el enfoque de mejora de los modelos creados, otras líneas futuras de investigación de este proyecto podría ser la implementación y comparación de distintos métodos de aprendizaje automático o análisis de regresión, realizando una evaluación y justificación de qué métodos funcionan mejor con este tipo de datos y por qué. Es una prueba que en principio estaba contemplada, aunque no a gran escala, en este proyecto pero no se ha podido llevar a cabo por tiempo, habiendo optado por centrarse el proyecto en una obtención de un modelo útil en vez de al estudio de los distintos métodos de realización. Otra línea de trabajo futura obvia que se realizará será la integración de las salidas de estos modelos con las \gls{pdf}s del marco de trabajo, objetivo último de este \gls{tfg} que no se ha llevado a cabo. Aunque esta integración a priori resulte fácil e intuitiva, es digna de estudio, ya que habrá que buscar una compensación entre ambas salidas e integrarlas con un filtro de partículas, según estaba previsto por su buen funcionamiento en el marco de trabajo previo. 
\\
\par Para finalizar, se concluye que se ha implementado una herramienta muy útil, fácilmente manejable y necesaria en la investigación a la que está enfocada para completar el uso de la información disponible de una manera óptima, además de estar abierta a líneas futuras tanto de investigaciones relacionadas como la mejora de esta misma. Se ha contrastado la eficiencia de \gls{rfr} para datos de este tipo y se corrobora la versatilidad de aplicaciones que puede tener esta herramienta dentro de la regresión y estimación de series temporales. 