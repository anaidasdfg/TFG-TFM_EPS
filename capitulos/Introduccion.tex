%%%%%%%%%%%%%%%%%%%%%%%%%%%%%%%%%%%%%%%%%%%%%%%%%%%%%%%%%%%%%%%%%%%%%%%%
% Plantilla TFG/TFM
% Escuela Politécnica Superior de la Universidad de Alicante
% Realizado por: Jose Manuel Requena Plens
% Contacto: info@jmrplens.com / Telegram:@jmrplens
%%%%%%%%%%%%%%%%%%%%%%%%%%%%%%%%%%%%%%%%%%%%%%%%%%%%%%%%%%%%%%%%%%%%%%%%

%\mychapter{1}{Capítulo 1. Introducción}
\chapter{Introducción}
\label{introduccion}
\par La telecomunicación se puede definir como toda transmisión y/o emisión y recepción de señales que representan signos, escritura, imágenes y sonidos o información de cualquier naturaleza por hilo, radioelectricidad, medios ópticos u otros sistemas electromagnéticos \cite{RAI}. Esto permite compartir información útil a distancia y engloba un amplio conjunto de sistemas y tecnologías. 
\\
\par En este apartado nos vamos a centrar en situarnos dentro de los distintos sistemas de telecomunicación, y más detenidamente en los relevantes para este proyecto. A continuación, se expondrán los objetivos concretos que se quieren alcanzar. Y, por último, cómo se va a organizar la memoria del proyecto. 

\section{Contexto}
\par Las telecomunicaciones forman parte de nuestro día a día y tienen cometidos de lo más variados: desde mandar un simple mensaje hasta comunicarse con una estación espacial, pero todos ellos engloban el manejo o el hecho de compartir información a distancia. 
\\
\par Dentro de los sistemas de telecomunicación encontramos el sistema de la teledetección, definido como la adquisición de información de un objeto, área o fenómeno, con instrumentos que no están en contacto directo con el objeto, según la \gls{rai} \cite{RAI}. Estos instrumentos van a medir la radiación electromagnética que emiten o reflejan los objetos observados. Algunos instrumentos pueden ser, por ejemplo, las cámaras fotográficas o los sistemas de radar (RAdio Detection And Ranging) o sonar.
\\
\par Las imágenes obtenidas desde satélite por sistemas radar son una gran fuente de información para aplicaciones de teledetección, como, por ejemplo, las predicciones meteorológicas, la realización de mapas topográficos o la monitorización de cultivos. Esta última, en la que se va a centrar este proyecto, requiere disponer de suficiente información periódica durante el tiempo que engloba el desarrollo completo del cultivo. En la monitorización de cultivos se encuentra la estimación del estado de los mismos, así como de variables descriptoras de este estado (biomasa, altura, etc.), que se obtienen a partir de imágenes que pueden ser analizadas de forma independiente o utilizando técnicas que aprovechen las series temporales de datos para la estimación. Estas técnicas son muy útiles en estos casos en los que la estimación va estrechamente ligada al transcurso del tiempo. 
\\
\par En la \gls{ua}, el grupo de investigación \gls{sst} ha diseñado un marco de trabajo sobre este tema basado en espacio de estados, que permite combinar de forma óptima los modelos de evolución esperable de los cultivos con los datos de otras fuentes como las imágenes \gls{sar} de satélite o la temperatura acumulada medida por una estación meteorológica. 
\\
\par Hasta la fecha, el modelo de observación utilizado que relacionaba las observaciones proporcionadas por las imágenes \gls{sar} con el estado fenológico de los cultivos era bastante simplificado y sus resultados no eran óptimos. Aquí entra el propósito de este \gls{tfg}, contribuir a su optimización mediante la generación de modelos de observación más complejos para las imágenes \gls{sar}. Estos modelos están basados en regresión con técnicas de machine learning que introduzcan la información de estas imágenes en el modelo de espacio de estados mencionado previamente. 

\section{Objetivos}
El objetivo general de este \gls{tfg} es estimar el estado de cultivos de arroz mediante el análisis series temporales con técnicas de aprendizaje automático y su unión a la línea de procesamiento original.
\\
\par Los objetivos concretos serían: 
\begin{itemize}
	\item Analizar las posibles técnicas de regresión de aprendizaje autónomo (por ejemplo, regresión con \gls{rf}) para estimar directamente el estado de los cultivos a partir de series temporales de datos. 
	\item Analizar las posibles técnicas de aprendizaje autónomo para ser combinados con algoritmos ya disponibles de dinámica de sistemas en la estimación del estado de cultivos.
	\item Incorporar dichas técnicas en la cadena de procesado disponible.
\end{itemize}

\section{Estructura de la memoria}
\par La estructura de la memoria se va a dividir en 3 capítulos principales las cuales son: marco teórico, metodología y resultados. Además de unas conclusiones finales valorando los resultados obtenidos.
\\
\par En el marco teórico se expondrá toda la teoría necesaria para la compresión de este proyecto en términos técnicos y dentro de un contexto y una investigación previa que este continúa. Veremos en él 3 secciones:
\begin{itemize}
	\item Teledetección, incluyendo cómo funcionan los sistemas radar, en concreto los \gls{sar}, qué información obtenemos de ellos en ciertos programas de satélites y las técnicas de detección que determinan cómo interpretar esta información.
	\item Técnicas de regresión y Machine Learning, donde se encuentra la clasificación de las distintas técnicas de \gls{ml}, algunos modelos y sus aplicaciones existentes, y, por último, el análisis mediante regresión. 
	\item Estimación de parámetros físicos de cultivos mediante
teledetección, donde se presentan la metodología general basada en el espacio de estados, los parámetros físicos de los cultivos con los que se puede trabajar y la estimación de estos mediante regresión. 
\end{itemize}

\par En cuanto a la metodología, se incluirán tanto las técnicas y métodos concretos que se van a utilizar, por qué motivos y qué esperamos obtener de ellos, como el software, el lenguaje de programación que vamos a emplear, las herramientas utilizadas y las bases de datos con las que vamos a trabajar, incluyendo su procedencia y procesamiento previo. 
\\
\par Por último, el apartado de resultados expondrá los resultados obtenidos con las diferentes técnicas de regresión y aprendizaje automático para los mismos datos. Estos resultados podrán ser fácilmente evaluados ya que se contrastarán, además, con los datos reales tomados en tierra de los mismos cultivos que se presentan en el dataset. 
