%%%%%%%%%%%%%%%%%%%%%%%%%%%%%%%%%%%%%%%%%%%%%%%%%%%%%%%%%%%%%%%%%%%%%%%%
% Plantilla TFG/TFM
% Escuela Politécnica Superior de la Universidad de Alicante
% Realizado por: Jose Manuel Requena Plens
% Contacto: info@jmrplens.com / Telegram:@jmrplens
%%%%%%%%%%%%%%%%%%%%%%%%%%%%%%%%%%%%%%%%%%%%%%%%%%%%%%%%%%%%%%%%%%%%%%%%

\chapter{Introducción}
\label{cap:introduccion}
\par La telecomunicación se puede definir como toda transmisión y/o emisión y recepción de señales que representan signos, escritura, imágenes y sonidos o información de cualquier naturaleza por hilo, radioelectricidad, medios ópticos u otros sistemas electromagnéticos \cite{RAI}. Esto permite compartir información útil a distancia y engloba un amplio conjunto de sistemas y tecnologías. 
\\
\par En este apartado nos vamos a centrar en situarnos dentro de los distintos sistemas de telecomunicación, y más detenidamente en los relevantes para este proyecto. A continuación, se expondrán los objetivos concretos que se quieren alcanzar. Y, por último, cómo se va a organizar la memoria del proyecto. 

\section{Contexto}
\par Las telecomunicaciones forman parte de nuestro día a día y tienen cometidos de lo más variados: desde mandar un simple mensaje hasta comunicarse con una estación espacial, pero todos ellos engloban el manejo o el hecho de compartir información a distancia. 

\subsection{Tecnología}
\par Dentro de los sistemas de telecomunicación encontramos la radio, la televisión, la telefonía fija y móvil, Internet por banda ancha o datos, la radionavegación o la teledetección. Todos ellos utilizan ondas electromagnéticas para sus comunicaciones, aunque estas se realicen mediante distintos medios de transmisión, que pueden ser guiados o no guiados, y con las modulaciones que se adapten a las necesidades de cada sistema. 
\\
\par Este proyecto se va a centrar en el sistema de la teledetección, definido como la adquisición de información un objeto, área o fenómeno, ya sea usando instrumentos de grabación o instrumentos de escaneo en tiempo real inalámbricos o que no están en contacto directo con el objeto, según la \gls{rai} \cite{RAI}. Estos instrumentos van a medir la radiación electromagnética que emiten o reflejan los objetos observados. Algunos de estos instrumentos pueden ser cámaras fotográficas, láseres, sistemas de radar o sonar, y pueden ser pasivos, miden la radiación natural emitida o reflejada, o activos, emiten energía que posteriormente será reflejada y detectada.
\\
\par Los instrumentos de medida, ya sean pasivos o activos, tienen la ventaja de poder estar situados a grandes distancias de la localización donde se quiera realizar la detección. Es por ello que se encuentran normalmente en satélites, aviones, barcos, etcétera, dependiendo de lo que se quiera medir. Las aplicaciones que engloba la teledetección son muy numerosas y suelen estar enfocadas a estudios científicos de ciertas áreas de la Tierra. 
\subsection{Caso particular}
\par Una vez introducida la tecnología existente para el área de este proyecto, concretamos cuál va a ser nuestra situación. 
\\
\par Ya que la aplicación en la que se mueve este proyecto es la agrícola, siendo esta la observación y adquisición de información de cultivos para su posterior estudio fenológico, la tecnología que se va a utilizar para ello son sistemas radar (radio detection and ranging), sistema activo, situado en un satélite artificial denominado Sentinel-1, del Programa Copérnico de la \gls{aema}. 
\\
\par En este área ya hay estudios previos que, a partir de datos similares se obtiene un estado de la fenología aproximado de los cultivos observados, como son 
\section{Objetivos}

\section{Estructura de la memoria}
\begin{equation}
	a_{0}
	\label{eq:max}
\end{equation}


